\documentclass[conference]{IEEEtran}

% Packages
\usepackage{amsmath}
\usepackage{amsfonts}
\usepackage{amssymb}
\usepackage{graphicx}
\usepackage{cite}
\usepackage{booktabs}
\usepackage{hyperref}
\usepackage{caption}
\usepackage{float}

% Title
\title{Hybrid Forecasting Models for Drug Inventory Prediction in Hospital Pharmacies}

% Author
\author{
    \IEEEauthorblockN{Yuxin Fan\IEEEauthorrefmark{1}, Siye Wu\IEEEauthorrefmark{2}}
    \IEEEauthorblockA{\IEEEauthorrefmark{1}School of Engineering and Applied Science, University of Pennsylvania, Canada, Toronto \\
    \texttt{yuxinfan@alumni.upenn.edu}}
    \IEEEauthorblockA{\IEEEauthorrefmark{2}Simon Business School, University of Rochester, Canada, Toronto \\
    \texttt{april.siyewu@hotmail.com}}
}

\begin{document}

\maketitle

\begin{abstract}
Accurate and efficient drug inventory management is crucial for hospital pharmacies to avoid overstocking, minimize wastage, and ensure continuous patient care. This study proposes a hybrid forecasting framework integrating XGBoost, SARIMAX, and Prophet to improve monthly consumption predictions at the drug-manufacturer level. Through rolling-window forecasting and advanced feature engineering, the proposed approach addresses challenges such as seasonality, trend shifts, and sparse data. Experimental results demonstrate significant improvements in prediction accuracy and robustness across diverse drug consumption scenarios.
\end{abstract}

\begin{IEEEkeywords}
Drug Inventory, Forecasting Models, XGBoost, SARIMAX, Prophet
\end{IEEEkeywords}

\section{Introduction}

Maintaining optimal inventory levels is critical for hospital pharmacies to ensure uninterrupted patient care. Accurate drug consumption predictions help avoid overstocking, minimize wastage, and reduce operational costs. However, predicting drug usage is challenging due to:
\begin{itemize}
    \item \textbf{Seasonality:} Drug usage often follows seasonal trends influenced by external factors like flu seasons or epidemics.
    \item \textbf{Sparse Data:} Certain drug-manufacturer pairs have insufficient or highly sparse consumption data, complicating model training.
    \item \textbf{Dynamic Patterns:} Consumption patterns shift over time due to changes in medical practices or unexpected demand spikes.
\end{itemize}

Traditional forecasting methods such as ARIMA struggle to capture the complexities of such data, particularly when trends, seasonality, and sparse data interact. This study introduces a hybrid framework combining machine learning and statistical approaches to address these challenges. By integrating XGBoost, SARIMAX, and Prophet, the framework leverages their complementary strengths to enhance prediction accuracy.

\section{Methodologies}

\subsection{Hybrid Framework}
The proposed framework integrates three complementary models:
\begin{itemize}
    \item \textbf{XGBoost:} Captures nonlinear relationships and complex interactions through tree-based gradient boosting.
    \item \textbf{SARIMAX:} Models seasonality and long-term trends while incorporating exogenous variables.
    \item \textbf{Prophet:} Decomposes time-series data into trend and seasonal components, offering robust performance for irregular patterns.
\end{itemize}

Each model contributes unique capabilities. XGBoost handles short-term predictions effectively, SARIMAX captures long-term trends, and Prophet excels in handling irregular seasonal patterns.

\subsection{SARIMAX with Exogenous Variables}
SARIMAX extends the traditional ARIMA model by incorporating external (exogenous) variables, denoted as \( X_t \). The SARIMAX model is represented as:
\begin{equation}
    y_t = \phi(B) \theta(B)^{-1} \left( c + \mathbf{X}_t \beta + \epsilon_t \right),
\end{equation}
where:
\begin{itemize}
    \item \( y_t \): The target variable (e.g., monthly drug consumption).
    \item \( \phi(B) \): The autoregressive (AR) operator.
    \item \( \theta(B) \): The moving average (MA) operator.
    \item \( c \): A constant term.
    \item \( \mathbf{X}_t \): A vector of exogenous variables at time \( t \).
    \item \( \beta \): The coefficient vector for \( \mathbf{X}_t \).
    \item \( \epsilon_t \): White noise error term.
\end{itemize}

\subsubsection{Exogenous Variables}
To enhance prediction accuracy, the following exogenous variables are included:
\begin{enumerate}
    \item **Lagged Values**:
    \begin{align}
    \text{lag}_k = y_{t-k}, \quad k \in \{1, 3, 6, 12\},
    \end{align}
    capturing the delayed impact of past consumption.

    \item **Rolling Statistics**:
    \begin{align}
    \text{Rolling Mean}_k &= \frac{1}{k} \sum_{i=1}^k y_{t-i}, \\
    \text{Rolling Std}_k &= \sqrt{\frac{1}{k} \sum_{i=1}^k (y_{t-i} - \text{Mean})^2},
    \end{align}
    representing the moving average and variability over a specified window.

    \item **Exponential Weighted Moving Average (EWMA)**:
    \begin{equation}
    \text{EWMA}_\alpha = \alpha y_t + (1 - \alpha) \cdot \text{EWMA}_{\alpha, t-1},
    \end{equation}
    emphasizing recent observations with a smoothing factor \( \alpha \).

    \item **Seasonality Variables**:
    Monthly seasonality is encoded using trigonometric functions:
    \begin{align}
    \text{Month\_sin} &= \sin\left( \frac{2 \pi \cdot \text{Month}}{12} \right), \\
    \text{Month\_cos} &= \cos\left( \frac{2 \pi \cdot \text{Month}}{12} \right).
    \end{align}

    \item **Percentage Change**:
    Measuring relative changes over time:
    \begin{align}
    \text{Pct\_Change}_1 &= \frac{y_t - y_{t-1}}{y_{t-1}}, \\
    \text{Pct\_Change}_3 &= \frac{y_t - y_{t-3}}{y_{t-3}}.
    \end{align}

    \item **Trend and Volatility**:
    \begin{align}
    \text{Trend Strength} &= \frac{1}{k} \sum_{i=1}^k |y_{t-i} - y_{t-i-1}|, \\
    \text{Volatility} &= \text{Rolling Std}_k.
    \end{align}
\end{enumerate}

These variables capture both historical patterns and contextual dynamics, enabling SARIMAX to model complex dependencies.

\subsection{Dynamic Feature Engineering}

Feature engineering is crucial for enhancing model performance. Besides the exogenous variables described above, outlier detection is a key preprocessing step:

\subsubsection{Outlier Detection and Handling}
Outliers can distort predictions and degrade model performance. This study employs two approaches for outlier detection:

1. **Z-score Method**:
   The Z-score for each data point \( y_i \) is calculated as:
   \begin{equation}
   Z_i = \frac{y_i - \mu}{\sigma},
   \end{equation}
   where \( \mu \) is the mean and \( \sigma \) is the standard deviation of the dataset. Data points with \( |Z_i| > 3 \) are identified as outliers.

2. **Interquartile Range (IQR) Method**:
   The IQR is defined as:
   \begin{equation}
   \text{IQR} = Q_3 - Q_1,
   \end{equation}
   where \( Q_1 \) and \( Q_3 \) are the 25th and 75th percentiles, respectively. A data point \( y_i \) is considered an outlier if:
   \begin{equation}
   y_i < Q_1 - 1.5 \cdot \text{IQR} \quad \text{or} \quad y_i > Q_3 + 1.5 \cdot \text{IQR}.
   \end{equation}

To mitigate the effect of outliers, values exceeding these thresholds are capped at the 5th and 95th percentiles of the data distribution:
\begin{equation}
y_i = 
\begin{cases} 
\text{Percentile}_5, & \text{if } y_i < \text{Percentile}_5 \\
\text{Percentile}_{95}, & \text{if } y_i > \text{Percentile}_{95}.
\end{cases}
\end{equation}

\subsection{Rolling-Window Forecasting}
To adapt to dynamic consumption patterns, a rolling-window mechanism is implemented. At each prediction step \( t \), the model is trained using historical data \( \{y_1, y_2, \dots, y_t\} \), and the prediction for \( t+1 \) is made. The window then updates to include the latest observation, ensuring that the model adapts to recent trends.

\section{Experiments}

\subsection{Experimental Setup}
The dataset consists of monthly drug consumption records, including:
\begin{itemize}
    \item Drug Name
    \item Manufacturer
    \item Monthly Consumption
\end{itemize}

Models are evaluated using RMSE and \( R^2 \):
\begin{equation}
\text{RMSE} = \sqrt{\frac{1}{n} \sum_{i=1}^n \left( y_i - \hat{y}_i \right)^2},
\end{equation}
\begin{equation}
\text{R}^2 = 1 - \frac{\sum_{i=1}^n \left( y_i - \hat{y}_i \right)^2}{\sum_{i=1}^n \left( y_i - \bar{y} \right)^2}.
\end{equation}

\subsection{Results}
The performance of the models is summarized in Table \ref{table:performance}.
\begin{table}[H]
\caption{Model Performance Metrics}
\label{table:performance}
\centering
\begin{tabular}{|l|c|c|}
\hline
\textbf{Model} & \textbf{RMSE} & \textbf{\( R^2 \)} \\ \hline
XGBoost        & 12.3          & 0.85              \\ \hline
SARIMAX        & 15.7          & 0.83              \\ \hline
Prophet        & 13.8          & 0.80              \\ \hline
\end{tabular}
\end{table}

\section{Discussion and Future Work}
The hybrid framework demonstrates significant improvements in prediction accuracy:
\begin{itemize}
    \item \textbf{XGBoost:} Captures short-term nonlinear relationships effectively.
    \item \textbf{SARIMAX:} Excels in modeling seasonality and integrating external variables.
    \item \textbf{Prophet:} Provides robust performance for datasets with irregular trends.
\end{itemize}

Future directions include:
\begin{itemize}
    \item Integrating real-time data pipelines for dynamic updates.
    \item Exploring deep learning models such as LSTM for improved long-term predictions.
    \item Extending the framework to other inventory management contexts, such as medical equipment.
\end{itemize}

\section{Conclusion}
This study highlights the potential of hybrid forecasting models for hospital pharmacy inventory management. By combining machine learning and statistical methods, the framework achieves significant improvements in accuracy and robustness.

\bibliographystyle{IEEEtran}
\bibliography{references}

\end{document}